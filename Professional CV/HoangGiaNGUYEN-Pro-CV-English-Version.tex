%%%%%%%%%%%%%%%%%
% This is an sample CV template created using altacv.cls
% (v1.1.5, 1 December 2018) written by LianTze Lim (liantze@gmail.com). Now compiles with pdfLaTeX, XeLaTeX and LuaLaTeX.
%
%% It may be distributed and/or modified under the
%% conditions of the LaTeX Project Public License, either version 1.3
%% of this license or (at your option) any later version.
%% The latest version of this license is in
%%    http://www.latex-project.org/lppl.txt
%% and version 1.3 or later is part of all distributions of LaTeX
%% version 2003/12/01 or later.
%%%%%%%%%%%%%%%%

%% If you need to pass whatever options to xcolor
\PassOptionsToPackage{dvipsnames}{xcolor}

%% If you are using \orcid or academicons
%% icons, make sure you have the academicons
%% option here, and compile with XeLaTeX
%% or LuaLaTeX.
% \documentclass[10pt,a4paper,academicons]{altacv}

%% Use the "normalphoto" option if you want a normal photo instead of cropped to a circle
% \documentclass[10pt,a4paper,normalphoto]{altacv}

\documentclass[10pt,a4paper,ragged2e]{altacv}

\usepackage{background}


%% AltaCV uses the fontawesome and academicon fonts
%% and packages.
%% See texdoc.net/pkg/fontawecome and http://texdoc.net/pkg/academicons for full list of symbols. You MUST compile with XeLaTeX or LuaLaTeX if you want to use academicons.

% Change the page layout if you need to
\geometry{left=1cm,right=9cm,marginparwidth=6.8cm,marginparsep=1.2cm,top=1.25cm,bottom=1.25cm}

% Change the font if you want to, depending on whether
% you're using pdflatex or xelatex/lualatex
\ifxetexorluatex
% If using xelatex or lualatex:
\setmainfont{Lato}
\else
% If using pdflatex:
\usepackage[utf8]{inputenc}
\usepackage[T1]{fontenc}
\usepackage[default]{lato}
\fi

\usepackage[hidelinks]{hyperref}

% Change the colours if you want to
\definecolor{Mulberry}{HTML}{72243D}
\definecolor{SlateGrey}{HTML}{2E2E2E}
\definecolor{LightGrey}{HTML}{666666}
%\colorlet{heading}{Sepia}
\colorlet{heading}{Orange}
%\colorlet{accent}{Mulberry}
\colorlet{accent}{Turquoise}
%\colorlet{emphasis}{SlateGrey}
\colorlet{emphasis}{MidnightBlue}
%\colorlet{body}{LightGrey}
\colorlet{body}{Black}

% Change the bullets for itemize and rating marker
% for \cvskill if you want to
\renewcommand{\itemmarker}{{\small\textbullet}}
\renewcommand{\ratingmarker}{\faCircle}

%% sample.bib contains your publications
\addbibresource{sample.bib}


\backgroundsetup{scale = 1., angle = 0, opacity = 1.,hshift=0.cm,
	contents = {\includegraphics[width = \paperwidth,
		height = \paperheight, keepaspectratio]
%		{CoverPage/Vextor-CV-20.pdf}}}
		{CoverPage/Abstracto-CV.pdf}}}

\begin{document}
	
	\name{Hoang Gia NGUYEN}
	%Frech Version
%	\tagline{Chercheur \& Développeur Sénior}
	%English Version
	\tagline{Researcher \& Senior Software Developer}
	
	%\photo{2.8cm}{Globe_High}
	\personalinfo{%
		% Not all of these are required!
		% You can add your own with \printinfo{symbol}{detail}
		\email{hoangia90@email.com}
		\phone{07 68 64 62 62}
		\mailaddress{Appartement 22, 3 allée Léon Blum, 59260, Hellemmes, Lille, France}
		\location{Lille, France}
		\homepage{https://lipn.univ-paris13.fr/\textasciitilde{}hoanggia.nguyen/}
		%  \twitter{@twitterhandle}
		\linkedin{https://www.linkedin.com/in/hoang-gia-nguyen/}
		\github{https://github.com/hoangia90}
		%% You MUST add the academicons option to \documentclass, then compile with LuaLaTeX or XeLaTeX, if you want to use \orcid or other academicons commands.
		% \orcid{orcid.org/0000-0000-0000-0000}
	}
	
	%% Make the header extend all the way to the right, if you want.
	\begin{fullwidth}
		\makecvheader
	\end{fullwidth}
	
	%% Depending on your tastes, you may want to make fonts of itemize environments slightly smaller
	% \AtBeginEnvironment{itemize}{\small}
	
	%% Provide the file name containing the sidebar contents as an optional parameter to \cvsection.
	%% You can always just use \marginpar{...} if you do
	%% not need to align the top of the contents to any
	%% \cvsection title in the "main" bar.
	
	%French Version
%	\cvsection[page1sidebar]{Expérience}
	%English Version
	\cvsection[page1sidebar]{Experience}
	
	%French Version
%	\cvevent{Chercheur - Post-Doctorat}{\href{https://www.inria.fr/fr}{INRIA}}{Nov 2018 -- En Cours}{Lille, France}
%	\begin{itemize}
%		\item Projet de Recherche : Design of Correct-by-Construction Self-Adaptive Cloud Applications Using Formal Methods
%		
%		\item Responsabilités : Chercheur, Développeur Full-stack Java \& Devops
%		
%		Développement et Déploiement de la Plate-forme \href{https://github.com/sbliudze/javabip-core}{JavaBIP} sur clouds: 
%		Concevoir et développer les modèles formels de micro-services en assurant sa coordination sur l'environnement multi-cloud (GCP, AWS, Azure, Heroku, etc.); Développer les mécanismes pour les systèmes de multi-cloud micro-services tels que les déploiement et les backup automatisés, le monitoring etc.  
%	\end{itemize}

	%English Version
	\cvevent{Researcher - Post-doctorate}{\href{https://www.inria.fr/fr}{INRIA}}{Nov 2018 -- Ongoing}{Lille, France}
	\begin{itemize}
		\item Research Project : Design of Correct-by-Construction Self-Adaptive Cloud Applications Using Formal Methods
		
		\item Responsibilities : Researcher, Java Full-stack Developer \& Devops
		
		Developing and Deploying \href{https://github.com/sbliudze/javabip-core}{JavaBIP} platform on clouds : 
		Designing and developing the formal models of micro-services while assuring the coordination in multi-cloud environment (GCP, AWS, Azure, Heroku, etc.); 
		Developing some mechanisms for multi-cloud micro-services systems such as auto deployment, backup, monitoring etc.  
	\end{itemize}

	
	
	\divider
	
	
	%French Version
%	\cvevent{Doctorant - Ph.D. \& Stagiaire}{\href{https://www.univ-paris13.fr/}{Université Sorbonne Paris Nord}}{2014 -- 2015 -- 2018}{Villetaneuse, France}
%	\begin{itemize}
%		\item Thématique : Efficient Parametric Verification of Parametric Timed Automata
%		\item Responsabilités : Chercheur, Développeur \href{https://ocaml.org/}{OCaml} \& \href{https://computing.llnl.gov/tutorials/mpi/}{MPI}
%		
%		Développement de la Plate-forme \href{https://www.imitator.fr/}{IMITATOR} : 
%		Proposer, concevoir, optimiser et réaliser les algorithmes  de synthèse de paramètres et ses versions distribuées sur des reseaux de supercomputers
%	\end{itemize}
	%English Version
	\cvevent{Ph.D. \& Internship}{\href{https://www.univ-paris13.fr/}{Université Sorbonne Paris Nord}}{2014 -- 2015 -- 2018}{Villetaneuse, France}
	\begin{itemize}
		\item Research Project : Efficient Parametric Verification of Parametric Timed Automata
		\item Responsibilities : Researcher, \href{https://ocaml.org/}{OCaml} \& \href{https://computing.llnl.gov/tutorials/mpi/}{MPI} Developer
		
		Developing \href{https://www.imitator.fr/}{IMITATOR} platform: 
		Proposing, designing, optimizing and implementing parameter synthesis algorithms and its distributed version running on networks of supercomputer. 
	\end{itemize}
	
	
	\divider
	
	
	%French Version
%	\cvevent{Enseignant}{\href{https://iutv.univ-paris13.fr/}{Université Sorbonne Paris Nord} \& Freelance}{2015 -- 2018}{Villetaneuse, France}
%	\begin{itemize}
%		\item Cours: Application Informatique Dédiée aux R\&T (Java), Bases de Données (Postgre), Bases des Services Réseaux
%		\item Cours (Freelance): Traitement automatique des langues (Python, Java, classification, clustering, .etc)
%	\end{itemize}
	%English Version
	\cvevent{Teaching}{\href{https://iutv.univ-paris13.fr/}{Université Sorbonne Paris Nord} \& Freelance}{2015 -- 2018}{Villetaneuse, France}
	\begin{itemize}
		\item Courses : Telecommunication and Network Application Programming (Java), Database Systems (Postgre), Networking Foundation
		\item Courses (Freelance) : Machine Learning, Natural Language Processing (Python, Java, classification, clustering, .etc)
	\end{itemize}

	
	\divider
	
%	\cvevent{Stagiaire}{\href{https://www.univ-paris13.fr/}{Université Sorbonne Paris Nord}}{2014 -- 2015}{Villetaneuse, France}
%	%\begin{itemize}
%	%	\item Thématique: Efficient Parametric Verification of Parametric Timed Automata.
%	%	\item Développement de la plate-forme \href{https://www.imitator.fr/}{IMITATOR}: Proposer, concevoir, optimiser et réaliser les algorithmes distribuées de synthèse de paramètres sur des reseaux de supercomputers. 
%	%\end{itemize}
%	
%	\divider
	
	%French Version
%	\cvevent{Membre}{\href{URL}{SAVE Lab - École Polytechnique Universitaire de Ho Chi Minh Ville}}{2014 -- 2015}{Ho Chi Minh, Vietnam}
%	\begin{itemize}
%		\item Responsabilité : Chercheur
%		
%		Travailler avec l'équipe de recherche SAVE en méthode formelle et AI; Aider à améliorer des algorithmes, des méthodes, des approches, des théories, etc. 
%	\end{itemize}
	%English Version
	\cvevent{Member}{\href{URL}{SAVE Lab - University of Technology of Ho Chi Minh City}}{2014 -- 2015}{Ho Chi Minh, Vietnam}
	\begin{itemize}
		\item Responsibility : Researcher
		
		Working with SAVE research team in formal methods and AI directions
	\end{itemize}

	
	\divider
	
	
	%French Version
%	\cvevent{Dévelopeur - à Mi-temps}{\href{http://hoangcuongelectric.vn/}{Hoang Cuong Electronic}}{2010 -- 2012}{Ho Chi Minh, Vietnam}
%	\begin{itemize}
%		\item Responsabilité : Dévelopeur (C\#, Microsoft SQL, \href{https://magento.com/}{Magento CMS})
%		
%		Développement Une Partie de Système ERP de l'Entreprise
%	\end{itemize}
	%English Version
	\cvevent{Developer - Part-time}{\href{http://hoangcuongelectric.vn/}{Hoang Cuong Electronic}}{2010 -- 2012}{Ho Chi Minh, Vietnam}
	\begin{itemize}
		\item Responsibility: Developer (C\#, Microsoft SQL, \href{https://magento.com/}{Magento CMS})
		
		Developing a Part of the ERP System of the Company
	\end{itemize}
	
	\clearpage
	
%	\medskip
	
	\cvsection[page2sidebar]{Software}
%	\cvsection[page2sidebar]{Logiciels}
	
	\cvevent{IMITATOR}{Parameter Synthesis for Real-Time Systems}
	{}{}
	\begin{itemize}
		\item \href{IMITATOR}{https://www.imitator.fr/}
	\end{itemize}
	
	\divider
	
	\cvevent{JavaBIP}{Coordination of concurrent Java components using a Java
		flavour of the BIP (Behaviour, Interactions, Priorities) framework
}
	{}{}
	\begin{itemize}
		\item \href{https://github.com/sbliudze}{https://github.com/sbliudze}
		\item \href{https://github.com/hoangia90}{https://github.com/hoangia90}
	\end{itemize}

	\divider
	
	%French Version
%	\cvevent{Autres Projets}{Mes nouveaux projets personnels seront mis en ligne sur GitHub}
%	{}{}
%	\begin{itemize}
%		\item \href{https://github.com/hoangia90}{https://github.com/hoangia90}
%	\end{itemize}
	%English Version
	\cvevent{Other Projects}{My new projects will be put on GitHub}
	{}{}
	\begin{itemize}
		\item \href{https://github.com/hoangia90}{https://github.com/hoangia90}
	\end{itemize}
	
	
	
	
	\cvsection{Publications}
	\nocite{*}
	
%	\printbibliography[heading=pubtype,title={\printinfo{\faBook}{Books}},type=book]
%	
%	\divider
%	
%	\printbibliography[heading=pubtype,title={\printinfo{\faFileTextO}{Journal Articles}},type=article]
%	
%	\divider
%	
%	\printbibliography[heading=pubtype,title={\printinfo{\faGroup}{Conference Proceedings}},type=inproceedings]
	
	\begin{itemize}
		\item Étienne André, Hoang Gia Nguyen et  Laure Petrucci. \textcolor{Turquoise}{\textbf{Distributed non-Zenoness parametric model checking (article de revues internationales)}}. Submitted.
%		Soumis.
		\item Hoang Gia Nguyen, Laure Petrucci, et Jaco van de Pol. \textcolor{Turquoise}{\textbf{Layered and Collecting NDFS with Subsumption for Parametric Timed Automata}}. 23nd International Conference on Engineering of Complex Computer Systems, IEEE CPS {ICECCS 2018}
		\item Étienne André, Hoang Gia Nguyen et Laure Petrucci. \textcolor{Turquoise}{\textbf{Efficient parameter synthesis using optimized state exploration strategies}}. 22nd International Conference on Engineering of Complex Computer Systems, IEEE CPS {ICECCS 2017}
		\item Étienne André, Hoang Gia Nguyen,  Laure Petrucci et Sun Jun. \textcolor{Turquoise}{\textbf{Parametric model checking timed automata under non-Zenoness assumption}}. 
		9$^{th}$ NASA Formal Methods Symposium {NFM 2017}
		\item Étienne André, Giuseppe Lipari, Hoang Gia Nguyen et Youcheng Sun. \textcolor{Turquoise}{\textbf{Reachability Preservation Based Parameter Synthesis for Timed Automata}}. 
		%In Klaus Havelund, Gerard Holzmann, Rajeev Joshi (eds.), 
		7$^{th}$ NASA Formal Methods Symposium {NFM 2015}
		%LNCS9058, Springer, pages 50–65, April 2015. Acceptance rate: 31\%. 
		\item Étienne André, Camille Coti et Hoang Gia Nguyen, \textcolor{Turquoise}{\textbf{Enhanced Distributed Behavioral Cartography of Parametric Timed Automata}}, 17$^{th}$ International Conference on Formal Engineering Methods ICFEM 2015
		
		%French Version
%		\item Pour voir plus de publications : \href{https://lipn.univ-paris13.fr/\textasciitilde{}hoanggia.nguyen/}{https://lipn.univ-paris13.fr/\textasciitilde{}hoanggia.nguyen/}
		%English Version
		\item For more publications : \href{https://lipn.univ-paris13.fr/\textasciitilde{}hoanggia.nguyen/}{https://lipn.univ-paris13.fr/\textasciitilde{}hoanggia.nguyen/}
	\end{itemize}
	
%	\medskip
%	
%	\cvsection{Intérêts}
%	
%	\begin{itemize}
%		\item Recherche: Model Checking et Verification, Calcul Distribué, Intelligence Artificielle, Big Data et Data Mining, Cloud Computing
%		\item Loisirs: Exploration de Nouvelles Technologies, Photographie, Natation, Randonnée, Lecture, Voyage

%	\end{itemize}
	
	
	\medskip
	
	
	%French Version
%	\cvsection{Ma Journée Habituelle}
%	
%	% Adapted from @Jake's answer from http://tex.stackexchange.com/a/82729/226
%	% \wheelchart{outer radius}{inner radius}{
%	% comma-separated list of value/text width/color/detail}
%	\wheelchart{1.5cm}{0.5cm}{%
%	  6/8em/accent!30/{Sommeil: \\Bon Sommeil ou Bon Cauchemar},
%	  3/8em/accent!40/Ma Famille ou Mes Amis(es),
%	  8/8em/accent!60/Travail,
%	  2/10em/accent/Sports et Relaxation,
%	  5/6em/accent!20/{Intérêts: \\Photographie, Lecture, Écrire sur mon Blog, etc.}
%	}
	%English Version
	\cvsection{A Day of My Life}
	
	% Adapted from @Jake's answer from http://tex.stackexchange.com/a/82729/226
	% \wheelchart{outer radius}{inner radius}{
	% comma-separated list of value/text width/color/detail}
	\wheelchart{1.5cm}{0.5cm}{%
		6/8em/accent!30/{Sleep: \\Good Sleep ou Good Nightmare},
		3/8em/accent!40/My Familly or My Friends,
		8/8em/accent!60/Work,
		2/10em/accent/Sport et Relaxation,
		5/6em/accent!20/{Interests: \\Photography, Reading Books, Blogging, etc.}
	}
	
	
%	\clearpage
%	\cvsection[page2sidebar]{Publications}
%	
%	\nocite{*}
%	
%	\printbibliography[heading=pubtype,title={\printinfo{\faBook}{Books}},type=book]
%	
%	\divider
%	
%	\printbibliography[heading=pubtype,title={\printinfo{\faFileTextO}{Journal Articles}},type=article]
%	
%	\divider
%	
%	\printbibliography[heading=pubtype,title={\printinfo{\faGroup}{Conference Proceedings}},type=inproceedings]
	
	%% If the NEXT page doesn't start with a \cvsection but you'd
	%% still like to add a sidebar, then use this command on THIS
	%% page to add it. The optional argument lets you pull up the
	%% sidebar a bit so that it looks aligned with the top of the
	%% main column.
	% \addnextpagesidebar[-1ex]{page3sidebar}
	
	
\end{document}
